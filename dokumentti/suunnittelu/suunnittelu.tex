\documentclass[gradu,emptyfirstpagenumber]{tktltiki}
\usepackage{url}
\usepackage{graphicx}

\begin{document}

\title{Hahmogeneraattori - Suunnitteludokumentti}
\author{Sami Saada - saada@cs.helsinki.fi}
\date{\today}
\level{Ohjelmoinnin harjoitusty�}

\maketitle

\doublespacing

\faculty{Matemaattis-luonnontieteellinen}
\department{Tietojenk�sittelytieteen laitos}
\depositeplace{}
\additionalinformation{}
\numberofpagesinformation{\numberofpages\ sivua}
\classification{}
\keywords{Ohjelmoinnin harjoitusty�, Hahmogeneraattori}

\begin{abstract}
Ohjelmoinnin harjoitusty�n, jonka aiheena on hahmogeneraattori, suunnitteludokumentti. Dokumentti sis�lt�� UML-kaaviot harjoitusty�st�.
\end{abstract}

\mytableofcontents

\section{Luokkarakenne}

Kuvat liitteen� (luokan\_rakenne.png, luokkakaavio.png).

\section{Luokkien rajapinnat}

Kuva liitteen� (luokka.png).

\section{K�ytt�liittym�suunnitelma}

Kuva liitteen� (gui.png). K�ytt�liittym� on tavallinen ikkuna, jossa valikkorivi ja vaihtuva sis�lt� keskell�.

\section{Toiminnan hahmottelu}

Kuva liitteen� (sekvenssikaavio.png). Sekvenssikaavio kuvaa hahmos��nn�n lataamista ja sen pohjalta uuden hahmon luomista ja tiedostoon tallentamista.

\lastpage
\appendices

\end{document}
