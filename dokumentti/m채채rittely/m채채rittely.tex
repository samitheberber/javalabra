\documentclass[gradu,emptyfirstpagenumber]{tktltiki}
\usepackage[latin1]{inputenc}
\usepackage[T1]{fontenc}
\usepackage[finnish]{babel}
\usepackage{times}

\begin{document}

\title{Määrittelydokumentti}
\author{Sami Saada - saada@cs.helsinki.fi}

\pagestyle{headings}

\maketitle

\chapter{Ohjelman tarkoitus ja yleiskuvaus}

Ohjelmani on hahmogeneraattori, jota käytetään roolipelihahmojen luontiin ja niillä pelaamiseen. Ohjelmaa voi käyttää pelinjohtajat sekä pelaajat. Pelinjohtaja voi etukäteen tehdä pelaajille hahmosäännöt tai hahmopohjat, joita pelaajat käyttävät luodessaan omat hahmonsa peliin. Valmiilla hahmolla voi pelata.

\subsection{Käyttötapaukset}

Pelinjohtaja tarkistaa oman hahmosääntönsä toimivuutta. Pelinjohtaja luo pelaajille valmiit hahmot. Pelinjohtaja luo ei-pelaajahahmon.

Pelaaja luo oman hahmonsa valmiin hahmosäännön pohjalta. Pelaaja tekee valmiista hahmosta henkilökohtaisen. Pelaaja pelaa hahmollaan. Pelaaja tallentaa hahmon tai lataa pelissä olevan hahmon.

\chapter{Rajoitukset}

Hahmosääntöjen ja hahmojen määrä on rajallinen levytilan kokoon nähden. Kerralla käsiteltävien hahmojen määrän rajaan 6 pelaajaan.

\chapter{dokkarin suunnittelua *poista*}

Harjoitustyön aiheena on hahmogeneraattori. Hahmogeneraattorilla pystyy luomaan hahmotyyppejä ja pelaamaan luoduilla hahmoilla.

Hahmoille pystyy luomaan eilaisia ominaisuuksia, joita kuvataan jollakin luvulla. Hahmoilla on erilaisia kykyjä, joihin vaikuttavat mm. hahmon tausta ja kokemus. Pelattavilla hahmoilla on henkilökohtaisia tietoja, kuten hahmon nimi, pelaajan nimi sekä hahmon sukupuoli.

Hahmopohjia voi ladata ja tallentaa. Myös pelissä olevia hahmoja voi ladata ja tallentaa. Tallennustilana toimii massamuistit.

Olennainen osa on hahmosääntöjen luonti. Hahmosääntö määrää sen, millaisia hahmoja voi luoda. Hahmosääntöjä voi ladata ja tallentaa. Tallennustilana toimii massamuistit.

\end{document}
